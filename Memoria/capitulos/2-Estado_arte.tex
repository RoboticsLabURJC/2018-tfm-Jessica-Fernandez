\chapter{Estado del arte}\label{cap.estado}

En este capítulo se revisará la bibliografía más relevante relacionada con la monitorización del tráfico obteniendo así información acerca de las diferentes técnicas propuestas. 

Nos centraremos en las técnicas propuestas para la detección, clasificación y seguimiento de vehículos. Comentaremos cuales son los métodos empleados, asi como sus ventajas e inconvenientes.

En los últimos años el precio de las cámaras se ha visto reducido y la potencia de cómputo de nuestros dispositivos ha aumentado, favoreciendo al desarrollo de sistemas de análisis de tráfico. Debido a esto, se trata de un área en continuo desarrollo desde los años 90, en el cual se pretende dar una solución en tiempo real. 

En la monitorización del tráfico hay que tener en cuenta tres puntos principales:
\begin{enumerate}
    \item Detección de vehículos
    \item Clasificación de vehículos
    \item Seguimiento de vehículos
\end{enumerate}

La detección consiste en localizar la posición de cada vehículo en la imagen. La clasificación consiste en identificar a que clase pertenece cada detección. Y el seguimiento consiste en asociar los vehículos en las sucesivas secuencias de video.

A continuación veremos las soluciones que plantean los diferentes autores para cada punto. Hay que decir que en muchas ocasiones la detección y la clasificación se hacen conjuntamente. No obstante vamos a comentar por separado cada fase.

\section{Detección de vehículos}

\section{Clasificación de vehículos}

\section{Seguimiento de vehículos}

