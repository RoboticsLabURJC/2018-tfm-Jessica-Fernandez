\chapter*{Resumen}

La gestión del tráfico rodado es un tema muy complejo pero con mucha importancia, pues tiene grandes repercusiones. Esta gestión típicamente la realizaba un operario encargado de extraer información de las imágenes que capturaba una cámara, la cual solo se centraba en obtener las imágenes. En los últimos años se está tratando de automatizar todas aquellas cosas que pueden realizarse sin necesidad de que el ser humano intervenga. Las cámaras son el sensor principal en la gestión del tráfico. Partiendo de esta premisa cada  vez se está tendiendo más a diseñar algoritmos capaces de extraer información automáticamente de dichas imágenes.\\

El objetivo de este trabajo es crear un sistema capaz de extraer información de las imágenes para monitorizar una carretera mediante una única cámara. Este sistema se llama \textit{Smart-Traffic-Sensor} y parte de un sistema previo denominado \textit{Traffic-Sensor}, en el cual se llevaba a cabo la monitorización del tráfico haciendo uso de técnicas clásicas de visión artificial, como sustracción de fondo, tracking, \acrshort{svm}, etc.\\

\textit{Smart-Traffic-Sensor} se centra en técnicas modernas como \textit{Deep Learning} para la detección y clasificación de vehículos. La monitorización se fundamenta en criterios  de  proximidad  espacial  y \acrfull{klt}. Para realizar la nueva versión neuronal del sistema, ha sido necesario crear una base de datos con suficiente información y recopilar una serie de videos donde poder evaluarlo.\\

El sistema se ha evaluado experimentalmente con diversos vídeos en condiciones diferentes con el fin de saber como se comportaría ante posibles cambios meteorológicos o en condiciones de baja calidad de imagen.




