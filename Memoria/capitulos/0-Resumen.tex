\chapter*{Resumen}

La gestión del tráfico es un tema muy complejo, pero com mucha importancia pues tiene grandes repercusiones. Esta gestión siempre se ha realizado de forma \textit{clásica}, mediante una cámara que se centraba únicamente en capturar imágenes y un operario que era el encargado de extraer información de estas. En las últimas décadas se está tratando de automatizar todas aquellas cosas que pueden realizarse sin necesidad de que el ser humano actúe tan activamente.

Las cámaras son el sensor principal en la gestión del tráfico, ya que son las que capturan las imágenes de donde posterormente se extraerá la información. Por tanto se puede decir que las imágenes son las que saben que está ocurriendo realmente en todo momento. Partiendo de esta premisa cada  vez se está tendiendo más a diseñar algoritmos capaces de extraer información automáticamente de dichas imágenes, pues como se ha dicho son las principales fuentes de información. Por supesto, esto puede extenderse a la monitorización de vehículos.

El objetivo de este trabajo es crear un sistema capaz de extraer información de las imágenes con el objetivo de llegar a monitorizar una carretera mediante una única cámara. Este sistema se llama \textit{Smart-Traffic-Sensor} y parte de un sistema previo denominado \textit{Traffic-Sensor}~\cite{traffic_monitor_redo}, en el cual se llevaba a cabo la monitorización del tráfico haciendo uso de técnicas clásicas.

\textit{Smart-Traffic-Sensor} se centra en técnicas modernas como \textit{Deep Learning} para la detección y clasificación de vehículos. La monitorización se fundamenta en criterios  de  proximidad  espacial  y \acrfull{klt}. Para poder realizar todo el proyecto ha sido necesario crear una base de datos con suficiente información y recopilar una serie de videos donde poder evaluarlo.

El sistema se ha evaluado con diversos videos en condiciones diferentes con el fin de saber como se comportaría ante posibles cambios meteorológicos o en condiciones de baja calidad de imagen.




